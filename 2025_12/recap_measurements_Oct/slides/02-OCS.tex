\section{OCS}
\subsection{Droplets}

\begin{frame}{OCS in Droplets - rotation direction change}
\begin{columns}[T,onlytextwidth]
  \column{.7\textwidth}
  \centering
  \includegraphics[width=\linewidth]{pic/ocs_droplets_3 (DA changed 17).png} % quadratische Datei
  % \captionof{figure}{Plot A}

  \column{.3\textwidth}
  \centering
  Information
  \begin{itemize}
    \item Changed DA position
    \begin{itemize}
      \item rotation direction changed
    \end{itemize}
  \end{itemize}
\end{columns}
\end{frame}

\begin{frame}{OCS in Droplets - oscillations nearly over the whole centrifuge}
\begin{columns}[T,onlytextwidth]
  \column{.7\textwidth}
  \centering
  \includegraphics[width=\linewidth]{pic/ocs_droplets_4 (DA changed 16.7).png} % quadratische Datei
  % \captionof{figure}{Plot A}

  \column{.3\textwidth}
  \centering
  Information
  \begin{itemize}
    \item good oscillations
  \end{itemize}
\end{columns}
\end{frame}

\begin{frame}{OCS in Droplets - where do the oscillation end?}
\begin{columns}[T,onlytextwidth]
  \column{.7\textwidth}
  \centering
  \includegraphics[width=\linewidth]{pic/ocs_droplets_5 (DA changed 16.6).png} % quadratische Datei
  % \captionof{figure}{Plot A}

  \column{.3\textwidth}
  \centering
  Information
  \begin{itemize}
    \item nearer look at the where the measurement stops following the rotation
  \end{itemize}
  \vspace{19pt}
  Questions
  \begin{itemize}
    \item How do we quantify the oscillations?
    \item How reproducible is this?
    \item Is the reduced contrast due to helium environment? (compare to jet) 
  \end{itemize}
\end{columns}
\end{frame}

\subsection{Jet}
\begin{frame}{OCS in Jet - begin of oscillations}
\begin{columns}[T,onlytextwidth]
  \column{.7\textwidth}
  \centering
  \includegraphics[width=\linewidth]{pic/ocs_jet_1.png} % quadratische Datei
  % \captionof{figure}{Plot A}

  \column{.3\textwidth}
  \centering
  Information
  \begin{itemize}
    \item start of rotations
    \item over low frequency ramp-up it looks similar to droplets
  \end{itemize}
\end{columns}
\end{frame}

\begin{frame}{OCS in Jet - end of oscillations}
\begin{columns}[T,onlytextwidth]
  \column{.7\textwidth}
  \centering
  \includegraphics[width=\linewidth]{pic/ocs_jet_2.png} % quadratische Datei
  % \captionof{figure}{Plot A}

  \column{.3\textwidth}
  \centering
  Information
  \begin{itemize}
    \item end of rotations
    \item same centrifuge as for the last slide
  \end{itemize}
  \vspace{19pt}
  Notes
  \begin{itemize}
    \item Safe to say that it is still spinning at 40 GHz
  \end{itemize}
\end{columns}
\end{frame}

\begin{frame}{OCS in Jet - very end to field free}
\begin{columns}[T,onlytextwidth]
  \column{.7\textwidth}
  \centering
  \includegraphics[width=\linewidth]{pic/ocs_jet_6 (DA changed 16.25).png} % quadratische Datei
  % \captionof{figure}{Plot A}

  \column{.3\textwidth}
  \centering
  Information
  \begin{itemize}
    \item changed starting frequency
    \begin{itemize}
      \item moved 40 GHz to the more intense part of the field profile
    \end{itemize}
  \end{itemize}
  \vspace{19pt}
  Questions
  \begin{itemize}
    \item Again: How we define if we still see oscillations?
  \end{itemize}
\end{columns}
\end{frame}