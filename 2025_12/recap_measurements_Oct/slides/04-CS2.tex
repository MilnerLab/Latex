\section{CS2}
% \usepackage{capt-of}  % nur nötig, wenn du Einzel-Captions willst
\subsection{Optimization}
\begin{frame}{Optimization of monomer conditions}
\begin{columns}[T,onlytextwidth]
  \column{.7\textwidth}
  \centering
  \includegraphics[width=\linewidth]{pic/CS2 optimization.png} % quadratische Datei
  % \captionof{figure}{Plot A}

  \column{.3\textwidth}
  \centering
  Information
  \begin{itemize}
    \item optimization the doping for different He droplet sizes
    \item ring size: 60 - 120 pixels
    \item polarization
    \begin{itemize}
      \item probe: vertical
      \item DA: horizontal
    \end{itemize}
  \end{itemize}
  \vspace{19pt}
  Notes
  \begin{itemize}
    \item not obvious which droplet size is ideal
    \item very sensitive to small VMI chamber pressure changed 
  \end{itemize}
\end{columns}
\end{frame}

\begin{frame}{Optimization of monomer conditions}
\begin{columns}[T,onlytextwidth]
  \column{.7\textwidth}
  \centering
  \includegraphics[width=\linewidth]{pic/CS2 dimer.png} % quadratische Datei
  \captionof{figure}{Sketches of possible (CS2)2 structures \footnote{\scriptsize
Pickering, J. D. (2017): \textit{Alignment and Imaging of Molecular Complexes Embedded in Helium Nanodroplets}. PhD thesis, Aarhus University.}
}

  \column{.3\textwidth}
  \centering
  Information
  \begin{itemize}
    \item doable because of difference in monomer/ dimer structure
    \begin{itemize}
      \item S+ ions only come along alignment pulse for monomers
    \end{itemize}
  \end{itemize}

\end{columns}
\end{frame}

\subsection{Droplets}
\begin{frame}{CS2 in Droplets - begin of centrifuge}
\begin{columns}[T,onlytextwidth]
  \column{.7\textwidth}
  \centering
  \includegraphics[width=\linewidth]{pic/cs2_droplets_1 (DA changed 16.7).png} % quadratische Datei
  % \captionof{figure}{Plot A}

  \column{.3\textwidth}
  \centering
  Information
  \begin{itemize}
    \item begin rotation
  \end{itemize}
  \vspace{19pt}
  Notes
  \begin{itemize}
    \item for low frequencies looks like all other molecules
  \end{itemize}
\end{columns}
\end{frame}

\begin{frame}{CS2 in Droplets - rotations at the centrifuge end}
\begin{columns}[T,onlytextwidth]
  \column{.7\textwidth}
  \centering
  \includegraphics[width=\linewidth]{pic/cs2_droplets_3 (DA changed 16.7).png} % quadratische Datei
  % \captionof{figure}{Plot A}

  \column{.3\textwidth}
  \centering
  Information
  \begin{itemize}
    \item end rotation
  \end{itemize}
  \vspace{19pt}
  Questions
  \begin{itemize}
    \item Is it still spinning at 20 GHz?
    \item What's the uncertainty of the CFG?
  \end{itemize}
\end{columns}
\end{frame}

\subsection{Jet}
\begin{frame}{CS2 in Jet - oscillations in gas phase}
\begin{columns}[T,onlytextwidth]
  \column{.7\textwidth}
  \centering
  \includegraphics[width=\linewidth]{pic/cs2_jet_1 (DA changed 16.7).png} % quadratische Datei
  % \captionof{figure}{Plot A}

  \column{.3\textwidth}
  \centering
  Information
  \begin{itemize}
    \item nice oscillations
    \item not exactly the same centrifuge as for droplets
  \end{itemize}
  \vspace{19pt}
  Notes
  \begin{itemize}
    \item Around 0-50 ps it's qualitatively the same as in droplets
  \end{itemize}
\end{columns}
\end{frame}

\begin{frame}{Cross Correlation - weird amplitude change}
\begin{columns}[T,onlytextwidth]
  \column{.7\textwidth}
  \centering
  \includegraphics[width=\linewidth]{pic/XCorr.png} % quadratische Datei
  % \captionof{figure}{Plot A}

  \column{.3\textwidth}
  \centering
  Information
  \begin{itemize}
    \item position of the cone can change depending on which arm of the centrifuge comes first
    \item never measured at the low frequencies  
  \end{itemize}
  \vspace{19pt}
  Notes
  \begin{itemize}
    \item could influence spinning of molecules for higher frequencies 
  \end{itemize}
\end{columns}
\end{frame}